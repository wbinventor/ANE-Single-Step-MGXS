%%%%%%%%%%%%%%%%%%%%%%%%%%%%%%%%%%%%%%%%%%%%%%%%%%%%%%%%%%%%%%%%%%%%%%%%%%%%%%%
\section{Conclusions}
\label{sec:conclusions}
%%%%%%%%%%%%%%%%%%%%%%%%%%%%%%%%%%%%%%%%%%%%%%%%%%%%%%%%%%%%%%%%%%%%%%%%%%%%%%%

Continuous energy Monte Carlo neutron transport methods have been increasingly applied for multi-group cross section generation. This paper introduced a single-step framework for MC-based MGXS generation for fine-mesh transport codes. The null and degenerate pin-wise spatial homogenization schemes were used to quantify the approximation error due to inter-pin spatial self-shielding effects in PWRs. The Local Neighbor Symmetry (LNS) pin-wise spatial homogenization was introduced to capture many of these self-shielding effects by homogenizing MGXS across groupings of fuel pins with similar neighboring spatial heterogeneities.

%The LNS scheme models spatial self-shielding with fewer materials than degenerate homogenization in order to accelerate the MC tally convergence rate by homogenizing MGXS across many fuel pins.

Two heterogeneous PWR benchmarks -- a fuel assembly and a 2$\times$2 assembly colorset with a water reflector -- were modeled in OpenMOC with MGXS generated by OpenMC. The single-step approach to MGXS generation was employed in which a single Monte Carlo eigenvalue calculation of the entire heterogeneous geometry was employed to collapse cross sections. Null spatial homogenization tallied MGXS for each fuel pin composition, while degenerate homogenization tallied MGXS for each unique fuel pin instance. LNS spatial homogenization applied a geometric template to group fuel pins with similar neighboring spatial heterogeneities. The eigenvalues, fission rates and U-238 capture rates predicted from multi-group calculations with OpenMOC were compared to a reference calculation with OpenMC for each of the three spatial homogenization schemes.

% The pin-wise U-238 capture rates, and to a lesser extent, the pin-wise fission rates, were better predicted when these effects were embedded into MGXS. 

The simulation results presented in this paper demonstrate that the eigenvalue is preserved when the same MC flux is used to collapse MGXS within a single-step MGXS generation scheme. The OpenMOC eigenvalues for each of the three pin-wise homogenization schemes are consistent to within 10 pcm. Hence, the choice of spatial homogenization technique is inconsequential to eigenvalue predictions. In contrast, the spatial homogenization scheme significantly impacts the accuracy of pin-wise reaction rate predictions. Non-negligible systematic approximation errors in the reaction rates arise when using MGXS which do not account for spatial self-shielding from neighboring fuel pins, control rod guide tubes and reflectors. Degenerate homogenization greatly reduced reaction rate errors with respect to null homogenization since it incorporated perturbations to the flux due to spatial heterogeneities. In particular, the maximum and mean U-238 capture rate errors were reduced by 2 -- 5$\times$ with degenerate homogenization. This demonstrated that the U-238 capture rate errors in deterministic multi-group transport calculations of PWRs are dominated by approximations to spatial self-shielding. In contrast, the fission rates were less sensitive to spatial self-shielding effects, and were only marginally reduced with degenerate homogenization.

LNS spatial homogenization performed as well as degenerate homogenization for a single assembly. However, the scheme failed to distinguish between pins at inter-assembly and assembly-reflector interfaces in the colorset, which resulted in a systematically large U-238 capture rate error for these fuel pins. The difficulty of generalizing the LNS algorithm for arbitrary geometries motivates the need for an unsupervised approach to accurately model spatial self-shielding effects. Such an approach should minimize the number of materials required to account for spatial self-shielding effects, and thus minimize the number of MC particle histories required to sufficiently converge MGXS statistical uncertainties.

%Although degenerate spatial homogenization was shown to be an effective approach to account for inter-pin spatial self-shielding, it is impractical for routine reactor analysis due to computational resource limitations. As a result of the fine-grained spatial tally mesh employed by degenerate homogenization, far more particle histories are needed to converge the MGXS tallies to obtain the same statistical uncertainties as with the simpler null scheme. Nevertheless, this analysis motivates the potential for a new spatial homogenization scheme as accurate as the degenerate scheme but requiring far fewer particle histories to converge MGXS.

%In general, the reaction rate errors for null homogenization are similar for groups of pins with similar neighboring heterogeneities. Hence, the errors may be equivalently reduced if an appropriate set of spatially self-shielded MGXS are defined for groups of pins with similar flux profiles. Future work should develop methods which best identify groups of pins to homogenize to achieve the accuracy of the degenerate scheme while simultaneously approaching the MC convergence rate of the null scheme. This is a topic of ongoing investigation and will be presented in future publications.
