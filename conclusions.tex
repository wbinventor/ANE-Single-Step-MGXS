%%%%%%%%%%%%%%%%%%%%%%%%%%%%%%%%%%%%%%%%%%%%%%%%%%%%%%%%%%%%%%%%%%%%%%%%%%%%%%%
\section{Conclusions}
\label{sec:conclusions}
%%%%%%%%%%%%%%%%%%%%%%%%%%%%%%%%%%%%%%%%%%%%%%%%%%%%%%%%%%%%%%%%%%%%%%%%%%%%%%%

% The eigenvalue is a key integral quantity used to assess the reactivity of a reactor. The fission rates are directly related to the relative power density of each fuel pin which is important for fuel depletion as well as thermal hydraulic feedback. The U-238 capture rates result in the production of Pu-239 which contributes up to 40\% of the power produced from fission in \ac{PWR}s at the end-of-life (EOL). Hence, the spatial distributions of the fission rates and U-238 capture rates must be correctly modeled for accurate high-fidelity depletion calculations. 
