%%%%%%%%%%%%%%%%%%%%%%%%%%%%%%%%%%%%%%%%%%%%%%%%%%%%%%%%%%%%%%%%%%%%%%%%%%%%%%%
\section{Conclusions}
\label{sec:conclusions}
%%%%%%%%%%%%%%%%%%%%%%%%%%%%%%%%%%%%%%%%%%%%%%%%%%%%%%%%%%%%%%%%%%%%%%%%%%%%%%%

-discuss continuous energy and multi-group transport options
  -recall intro: must remember need to both isolate and quantify multi-group approximation errors in order to resolve them in new fancy fine-mesh transport codes
  
This paper isolated and quantified the impact of capturing inter-pin spatial self-shielding effects in MGXS for fuel pins in PWR geometries. Two heterogeneous PWR benchmarks -- a fuel assembly and a 2$\times$2 assembly colorset with reflector -- were modeled in OpenMOC with MGXS generated by OpenMC. A single-step approach to MGXS generation was employed in which a single Monte Carlo eigenvalue calculation of the entire heterogeneous geometry was employed to collapse cross sections. Two spatial homogenization schemes were introduced to enable a direct quantification of spatial self-shielding effects from local heterogeneities in MGXS. Null spatial homogenization tallied MGXS for each fuel pin composition, while degenerate homogenization tallied MGXS for each unique fuel pin instance.

% The pin-wise U-238 capture rates, and to a lesser extent, the pin-wise fission rates, were better predicted when these effects were embedded into MGXS. 

The results presented in this paper show that non-negligible systematic approximation errors in the reaction rates arise when using MGXS which do not account for spatial self-shielding from neighboring fuel pins, control rod guide tubes, burnable poisons and reflectors. Degenerate homogenization greatly reduced reaction rate errors with respect to null homogenization since it incorporated perturbations to the flux due to spatial heterogeneities. In particular, the maximum and mean U-238 capture rate errors were reduced by 2 -- 5$\times$ with degenerate homogenization. In contrast, the fission rates were less sensitive to spatial self-shielding effects, and were only marginally reduced with degenerate homogenization.

%This paper also illustrates that when the same MC flux is used to collapse MGXS within a single-step MGXS generation scheme is employed -- such as either null or degenerate homogenization -- global reactivity will always be preserved. In particular, the OpenMOC eigenvalues match OpenMC to within nearly 100 \ac{pcm} for both benchmarks and homogenization schemes with 70-group MGXS. Therefore, although spatial self-shielding effects must be accounted for in MGXS for accurate pin-wise reaction rate predictions, the spatial homogenization technique is inconsequential to eigenvalue predictions.

Although degenerate spatial homogenization was shown to be an effective approach to account for inter-pin spatial self-shielding, it is impractical for routine reactor analysis due to computational resource limitations. As a result of the fine-grained spatial tally mesh employed by degenerate homogenization, far more particle histories are needed to converge the MGXS tallies to obtain the same statistical uncertainties as with the simpler null scheme. Nevertheless, this analysis motivates the potential for a new spatial homogenization scheme as accurate as the degenerate scheme but requiring far fewer particle histories to converge MGXS.

In general, the reaction rate errors for null homogenization are similar for groups of pins with similar neighboring heterogeneities. Hence, the errors may be equivalently reduced if an appropriate set of spatially self-shielded MGXS are defined for groups of pins with similar flux profiles. Future work should develop methods which best identify groups of pins to homogenize to achieve the accuracy of the degenerate scheme while simultaneously approaching the MC convergence rate of the null scheme. This is a topic of ongoing investigation and will be presented in future publications.