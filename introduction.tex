%%%%%%%%%%%%%%%%%%%%%%%%%%%%%%%%%%%%%%%%%%%%%%%%%%%%%%%%%%%%%%%%%%%%%%%%%%%%%%%
\section{Introduction}
\label{sec:intro}
%%%%%%%%%%%%%%%%%%%%%%%%%%%%%%%%%%%%%%%%%%%%%%%%%%%%%%%%%%%%%%%%%%%%%%%%%%%%%%%

The nuclear reactor physics community has long strived for deterministic neutron transport-based tools for whole-core reactor analysis. A key challenge for whole-core multi-group transport methods is accurate reactor agnostic multi-group cross section (MGXS) generation. The MGXS generation process applies a series of approximations to produce spatially homogenized and energy condensed MGXS in each spatial zone and energy group. ... However, the practical impact of ... is less understood. This paper investigates the ... and quantifies its significance for heterogeneous PWR problems.

This work employs Monte Carlo (MC) neutron transport simulations to generate MGXS. Monte Carlo methods have increasingly been used to generate few group constants for coarse mesh diffusion, most notably by the Serpent MC code \citep{serpent2013manual}, and to a much lesser extent, for high-fidelity neutron transport methods~\citep{redmond1997multigroup, nelson2014improved, cai2014condensation, boyd2016thesis}. The advantage of a MC-based approach is that all of the relevant physics are directly embedded into MGXS by weighting the continuous energy cross sections with a statistical proxy to the ``true'' neutron flux. 

This paper seeks to identify the bias between continuous energy and multi-group transport methods for MGXS libraries which account for spatial self-shielding effects\footnote{The effects of neighboring pins, burnable poisons, reflectors and the core baffle are each of interest in the context of spatial self-shielding.} to varying degrees. In particular, this paper quantifies the difference in the approximation error between simulations in which the same MGXS are used in each unique fuel pin (\textit{e.g.}, each fuel enrichment) and those in which unique MGXS are used in each and every pin. The former case does little if anything to model spatial self-shielding effects, whereas the latter case ``fully'' resolves these effects, albeit at the expense of very large MGXS libraries.

% This difference in approximation error motivates the development of a novel methodology in the following chapters which uses statistical clustering to capture spatial self-shielding effects in MGXS.

The content in this paper is organized as follows. Two different schemes for spatial homogenization of pin-wise MGXS are introduced in~\autoref{sec:methodology}. The need for a new, more flexible and specialized approach to spatial homogenization which appropriately captures spatial self-shielding effects with minimal computational expense is discussed in~\autoref{sec:conclusions}.