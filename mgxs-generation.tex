%%%%%%%%%%%%%%%%%%%%%%%%%%%%%%%%%%%%%%%%%%%%%%%%%%%%%%%%%%%%%%%%%%%%%%%%%%%%%%%
\section{MGXS Generation}
\label{sec:mgxs-generation}
%%%%%%%%%%%%%%%%%%%%%%%%%%%%%%%%%%%%%%%%%%%%%%%%%%%%%%%%%%%%%%%%%%%%%%%%%%%%%%%

first paragraph: MC for MGXS generation
-

second paragraph: outline section
-d

Deterministic methods typically do not make use of continuous energy cross section data and instead discretize the energy domain through the multi-group energy approximation. The multi-group approximation requires an \textit{a priori} estimate of the neutron flux to compute the multi-group cross sections in each energy group and spatial zone.

%The multi-group approximation considerably reduces the necessary dataset footprint, and enables deterministic methods to be more effectively formulated for vectorization and cache reuse than MC methods. 


%%%%%%%%%%%%%%%%%%%%%%%%%%%%%%%%%%%%%%%%%%%%%%%%%%%%%%%%%%%%%%%%%%%%%%%%%%%%%%%
\subsection{A Single-Step Framework}
\label{subsec:single-level}

In general, MGXS generation schemes use a multi-step approach to decouple the energy, angular and spatial dimensions of the transport equation. The multi-step approach typically applies high-fidelity models of the energy self-shielding physics to low-fidelity geometric models of unique core components. The multi-step approach uses a combination of models of varying complexity to optimize overall simulation speed with accuracy. However, this is often done at the expense of generality. For example, multi-step MGXS generation schemes do not typically model inter-assembly physics or the effect of reflectors and other core heterogeneities on the spatial distribution of the flux. Instead, geometric heuristics are often used to embed spatial self-shielding effects in MGXS for similarly shielded spatial zones (\textit{e.g.}, fuel pins with similar neighboring pins). The approximations to the energy and spatial variation of the flux introduce approximation error in full-core calculations and limit the core design parameter space for which multi-level schemes may be applied. 

This work abandons the multi-step approach in favor of a single-step framework. Furthermore, this work employs Monte Carlo methods to generate MGXS since it presents a natural approach to replace engineering prescriptions to approximate the flux with a stochastic approximation of the exact flux. However, MC-based MGXS generation methods to date have retained the multi-step geometric framework to tabulate MGXS for individual reactor components -- such as infinite fuel pins and/or assemblies -- for subsequent use in full-core multi-group calculations. Although the use of MC within a multi-step framework eliminates the need to approximate the flux in energy, it does not account for spatial self-shielding effects throughout a reactor core. 

This work takes an additional step and uses MC eigenvalue simulations of the complete heterogeneous geometry to simultaneously account for all energy and spatial effects in a single step. The single-step framework may be impractical for MGXS generation for industrial applications since it is constrained by the slow convergence rate of Monte Carlo tallies. Nevertheless, it allows for the rigorous quantification of approximation error due to spatial self-shielding models used to generate MGXS, which is the focal point of this paper.


%%%%%%%%%%%%%%%%%%%%%%%%%%%%%%%%%%%%%%%%%%%%%%%%%%%%%%%%%%%%%%%%%%%%%%%%%%%%%%%
\subsection{Pin-wise Spatial Self-Shielding Models}
\label{subsec:homogenize}

-single level Monte Carlo calculation
-focus less on introducing these as ``schemes'' per se
  -rather two spatial self-shielding models to quantify approx. error

This paper employs two different spatial homogenization schemes to model spatial self-shielding effects in MGXS. Although all spatial zones may experience spatial self-shielding, this chapter only models the impact of spatial self-shielding on MGXS in fissile regions. The null and degenerate spatial homogenization schemes are introduced in~\autoref{subsubsec:homogenize-null} and~\autoref{subsubsec:homogenize-degenerate}, respectively. These schemes model spatial self-shielding for each fuel pin with increasing granularity and complexity. 

%The \texttt{openmc.mgxs} module was used to compute 70-group MGXS with OpenMC for both the assembly and colorset benchmarks. 

%%%%%%%%%%%%%%%%%%%%%%%%%%%%%%%%%%%%%%%%%%%%%%%%%%%%%%%%%%%%%%%%%%%%%%%%%%%%%%%
\subsubsection{Null Homogenization}
\label{subsubsec:homogenize-null}

The \textit{null} spatial homogenization scheme uses a single Monte Carlo calculation of the complete heterogeneous geometry to generate MGXS for each material. In this way, the null scheme fully abandons the multi-level approach used by most traditional approaches to generate MGXS. The spatially self-shielded flux is used to collapse the cross sections in each material with a unique isotopic composition. The null scheme does not account for spatial self-shielding effects experienced by different fuel pins filled by the same type of fuel, and instead averages these effects across the entire geometry. A single MGXS is employed in each instance of a material zone, such as a fuel pin replicated many times throughout a benchmark geometry.

%%%%%%%%%%%%%%%%%%%%%%%%%%%%%%%%%%%%%%%%%%%%%%%%%%%%%%%%%%%%%%%%%%%%%%%%%%%%%%%
\subsubsection{Degenerate Homogenization}
\label{subsubsec:homogenize-degenerate}

Unlike the null scheme, the \textit{degenerate} scheme accounts for the different spatial self-shielding effects experienced by each instance of each fuel pin throughout a heterogeneous geometry. Like the null scheme, a single MC calculation of the complete heterogeneous geometry is used to generate MGXS for all materials. Unlike the null scheme, the MGXS are tallied separately for each instance of fissile material zones. For example, if a heterogeneous benchmark includes $N$ fuel pins, then $N$ collections of MGXS are separately tabulated for each fuel pin instance. The degenerate scheme tallies different MGXS even if the isotopic compositions in the fuel pin instances are identical (\textit{e.g.}, fresh fuel at the beginning of life) since each instance may experience different spatial self-shielding effects and hence have different MGXS.

The degenerate scheme generates MGXS for each fuel pin instance using OpenMC's distributed cell tallies~\citep{lax2014distribcell}. The OpenCG region differentiation algorithm~\citep{boyd2015opencg} is used to build a new OpenMOC geometry with unique cells and materials for each fuel pin. The MGXS are appropriately selected from OpenMC's distributed cell tallies to populate the MGXS in the OpenMOC materials. Multi-group transport calculations with MGXS generated using null and degenerate schemes may be compared to quantify the impact of modeling spatial self-shielding effects in MGXS for fissile zones in heterogeneous geometries.