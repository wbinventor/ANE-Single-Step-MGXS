%%%%%%%%%%%%%%%%%%%%%%%%%%%%%%%%%%%%%%%%%%%%%%%%%%%%%%%%%%%%%%%%%%%%%%%%%%%%%%%
\section{Simulation Tools}
\label{sec:simulation-tools}
%%%%%%%%%%%%%%%%%%%%%%%%%%%%%%%%%%%%%%%%%%%%%%%%%%%%%%%%%%%%%%%%%%%%%%%%%%%%%%%

-this should probably be written after the intro???

first paragraph: outline section
-motivate methodology by mission behind paper
  -quantify approx error due to MGXS spatial self-shielding
-uses MC to generate MGXS 
  -uses single level

%This paper investigates Monte Carlo as a means to generate multi-group cross sections for fine-mesh transport codes. 

This work required the development of a ``simulation triad'' encompassing three primary simulation tools. First, the OpenMC Monte Carlo code~\citep{romano2013openmc} was utilized to generate multi-group cross sections. Second, the MGXS were used by the OpenMOC code~\citep{boyd2014openmoc} for deterministic multi-group transport calculations. Third, the OpenCG library~\citep{boyd2015opencg} enabled the processing and transfer of tally data on combinatorial geometry (CG) meshes between OpenMC and OpenMOC. Each of the OpenMC, OpenMOC, and OpenCG codes is highlighted in~\autoref{subsec:openmc},~\autoref{subsec:openmoc} and~\autoref{subsec:opencg}.

-outline sections for spatial homogenization?? probably needs further motivating

%In addition, a significant amount of infrastructural code was developed to process the results produced by OpenMC and OpenMOC.


%%%%%%%%%%%%%%%%%%%%%%%%%%%%%%%%%%%%%%%%%%%%%%%%%%%%%%%%%%%%%%%%%%%%%%%%%%%%%%%
\subsection{Continuous Energy Calculations with OpenMC}
\label{subsec:openmc}

The OpenMC continuous energy Monte Carlo (MC) code \citep{romano2013openmc} was employed to generate multi-group cross sections, and reference eigenvalues and pin-wise fission and capture reaction rates. The \texttt{openmc.mgxs} Python module was used to tally multi-group cross sections in CASMO's seventy energy group structure~\citep{rhodes2006casmo} from a single eigenvalue calculation. The multi-group cross sections were calculated with OpenMC's distributed cell tally algorithm~\citep{lax2014distribcell}, which permits spatial tally zones across repeated cell instances. In particular, unique MGXS were computed for each fuel pin cell with distributed cell tallies in the repeating lattice benchmarks described in~\autoref{sec:test-cases}. The OpenMC simulations were performed with 1000 batches with 10$^{6}$ particle histories per batch for each benchmark. Stationarity of the fission source was obtained with 100 inactive batches for each benchmark.

The OpenMC simulations used the ``iso-in-lab'' feature to enforce isotropic in lab scattering. The ``iso-in-lab'' feature samples the outgoing neutron energy from the scattering laws prescribed by the continuous energy cross section library, but the outgoing neutron direction of motion is sampled from an isotropic in lab distribution. Although isotropic in lab scattering is a poor approximation for LWRs, it eliminated scattering source anisotropy as one possible cause of approximation error between OpenMC and OpenMOC. This simplification made it possible to isolate the approximation error resulting from the spatial self-shielding model used to generate MGXS, which is the focal point of this paper.

The degenerate scheme generates MGXS for each fuel pin instance using OpenMC's distributed cell tallies~\citep{lax2014distribcell}. The OpenCG region differentiation algorithm~\citep{boyd2015opencg} is used to build a new OpenMOC geometry with unique cells and materials for each fuel pin. The MGXS are appropriately selected from OpenMC's distributed cell tallies to populate the MGXS in the OpenMOC materials. Multi-group transport calculations with MGXS generated using null and degenerate schemes may be compared to quantify the impact of modeling spatial self-shielding effects in MGXS for fissile zones in heterogeneous geometries.

%%%%%%%%%%%%%%%%%%%%%%%%%%%%%%%%%%%%%%%%%%%%%%%%%%%%%%%%%%%%%%%%%%%%%%%%%%%%%%%
\subsection{Multi-Group Calculations with OpenMOC}
\label{subsec:openmoc}

The OpenMOC code~\citep{boyd2014openmoc} was employed to use the MGXS generated by OpenMC for deterministic multi-group calculations. The OpenMOC code is a 2D method of characteristics code designed for fixed source and eigenvalue neutron transport calculations. OpenMOC approximates the scattering source as isotropic in the lab coordinate system, and discretizes the geometry into flat source regions (FSRs) which approximate the neutron source as constant across each spatial zone. The OpenMOC eigenvalue and energy-integrated, pin-wise reaction rates were compared with the reference solution computed by OpenMC.

Each OpenMOC simulation used a characteristic track laydown with 128 azimuthal angles and 0.05 cm spacing. All eigenvalue calculations were converged to 10$^{-5}$ on the root mean square of the energy-integrated fission source in each FSR. The Coarse Mesh Finite Difference (CMFD) acceleration scheme was employed on a pin-wise spatial mesh to reduce the number of iterations required to converge the fine-mesh transport calculations. The 70-group MGXS used for MOC were collapsed to a 14-group structure for CMFD to significantly improve the speed of the CMFD eigenvalue calculations.